\documentclass[11pt,a4paper]{article}
\usepackage[utf8]{inputenc}
\usepackage[T1]{fontenc}
\usepackage[french]{babel}
\usepackage{hyperref}
\usepackage{booktabs}
\usepackage{longtable}
\usepackage{geometry}
\usepackage{xcolor}
\usepackage{titlesec}

\geometry{margin=2.5cm}

\hypersetup{
    colorlinks=true,
    linkcolor=blue,
    urlcolor=blue,
    citecolor=blue
}

\titleformat{\section}{\Large\bfseries}{\thesection}{1em}{}
\titleformat{\subsection}{\large\bfseries}{\thesubsection}{1em}{}

\title{\textbf{Bibliographie sur l'Hypersensibilité aux Faibles Doses de Radiation (HRS/IRR)}\\[0.5cm]
\large Publications Fondatrices et Historiques (1983--2003)\\
Publications Récentes (2022--2025)}
\author{Compilation bibliographique}
\date{Décembre 2025}

\begin{document}

\maketitle

\tableofcontents
\newpage

%==============================================================================
\section{Introduction}
%==============================================================================

L'hyper-radiosensibilité aux faibles doses (HRS -- \textit{Hyper-RadioseSensitivity}) est un phénomène par lequel les cellules meurent d'une sensibilité excessive à de petites doses uniques de rayonnement ionisant (typiquement $<$0.3 Gy), suivie d'une radioresistance induite (IRR -- \textit{Induced RadioResistance}) à des doses légèrement plus élevées. Ce document compile les publications fondatrices et récentes dans ce domaine.

%==============================================================================
\section{Publications Fondatrices et Historiques (1983--2003)}
%==============================================================================

%------------------------------------------------------------------------------
\subsection{Études in vivo précurseurs (années 1980)}
%------------------------------------------------------------------------------

\begin{enumerate}

\item \textbf{Joiner MC, Maughan RL, Fowler JF, Denekamp J (1983)}\\
\textit{The RBE for mouse skin irradiated with 3-MeV neutrons: single and fractionated doses.}\\
Radiation Research, 95(1):130-141.\\
PMID: 6878624\\
\url{https://pubmed.ncbi.nlm.nih.gov/6878624/}

\item \textbf{Denekamp J, Joiner MC, Maughan RL (1984)}\\
\textit{Neutron RBEs for mouse skin at low doses per fraction.}\\
Radiation Research, 98(2):317-331.\\
PMID: 6729041\\
\url{https://pubmed.ncbi.nlm.nih.gov/6729041/}

\item \textbf{Stewart FA, Luts A, Lebesque JV (1984)}\\
\textit{The RBE for renal damage after irradiation with 3 MeV neutrons.}\\
British Journal of Radiology.\\
PMID: 6442969\\
\url{https://pubmed.ncbi.nlm.nih.gov/6442969/}\\
\textcolor{gray}{\textit{Note: Premiers indices que les RBE élevés peuvent être observés dans les tissus à renouvellement lent après de faibles doses par fraction.}}

\item \textbf{Joiner MC, Denekamp J, Maughan RL (1986)}\\
\textit{The use of top-up experiments to investigate the effect of very small doses per fraction in mouse skin.}\\
International Journal of Radiation Biology, 49:565-580.

\item \textbf{Joiner MC, Johns H (1988)} $\bigstar$ \textbf{ÉTUDE FONDATRICE}\\
\textit{Renal damage in the mouse: the response to very small doses per fraction.}\\
Radiation Research, 114(2):385-398.\\
PMID: 3375433\\
\url{https://pubmed.ncbi.nlm.nih.gov/3375433/}\\
\textcolor{gray}{\textit{Note: Expériences utilisant des doses de rayons X de 0.2 à 1.6 Gy par fraction et des neutrons de 0.05 à 0.25 Gy par fraction sur les reins de souris.}}

\end{enumerate}

%------------------------------------------------------------------------------
\subsection{Premières études in vitro sur lignées cellulaires (1993--1997)}
%------------------------------------------------------------------------------

\begin{enumerate}
\setcounter{enumi}{5}

\item \textbf{Marples B, Joiner MC (1993)} $\bigstar$ \textbf{PREMIÈRE DÉMONSTRATION IN VITRO}\\
\textit{The response of Chinese hamster V79 cells to low radiation doses: evidence of enhanced sensitivity of the whole cell population.}\\
Radiation Research, 133(1):41-51.\\
PMID: 8434112\\
\url{https://pubmed.ncbi.nlm.nih.gov/8434112/}\\
\textcolor{gray}{\textit{Note: Mesures haute résolution de la survie des cellules V79-379A après doses uniques de rayons X (0.01--10.0 Gy). L'effet par unité de dose a augmenté d'un facteur $\sim$2, passant de 0.19 Gy$^{-1}$ à 1 Gy à 0.37 Gy$^{-1}$ à 0.1 Gy.}}

\item \textbf{Lambin P, Marples B, Fertil B, Malaise EP, Joiner MC (1993)}\\
\textit{Hypersensitivity of a human tumour cell line to very low radiation doses.}\\
International Journal of Radiation Biology, 63:639-650.\\
PMID: 8099110\\
\url{https://pubmed.ncbi.nlm.nih.gov/8099110/}

\item \textbf{Malaise EP, Lambin P, Joiner MC (1994)}\\
\textit{Radiosensitivity of human cell lines to small doses. Are there some clinical implications?}\\
Radiation Research, 138(1 Suppl):S25-27.\\
PMID: 8146319\\
\url{https://pubmed.ncbi.nlm.nih.gov/8146319/}\\
\textcolor{gray}{\textit{Note: Revue utilisant cytométrie de flux et DMIPS montrant l'hypersensibilité à très faibles doses ($<$0.5 Gy) suivie d'une augmentation de radioresistance.}}

\item \textbf{Lambin P, Fertil B, Malaise EP, Joiner MC (1994)}\\
\textit{Multiphasic Survival Curves for Cells of Human Tumor Cell Lines: Induced Repair or Hypersensitive Subpopulation?}\\
Radiation Research, 138(1 Suppl):S32-S36.\\
PMID: 8146321\\
\url{https://www.jstor.org/stable/3578756}

\item \textbf{Wouters BG, Skarsgard LD (1994)}\\
\textit{The response of a human tumor cell line to low radiation doses: Evidence of enhanced sensitivity.}\\
Radiation Research, 138(1 Suppl):S76-S80.\\
PMID: 8146333\\
\url{https://pubmed.ncbi.nlm.nih.gov/8146333/}

\item \textbf{Marples B, Joiner MC (1995)}\\
\textit{The elimination of low-dose hypersensitivity in Chinese hamster V79-379A cells by pretreatment with X rays or hydrogen peroxide.}\\
Radiation Research, 141(2):160-169.\\
PMID: 7838950\\
\url{https://pubmed.ncbi.nlm.nih.gov/7838950/}

\item \textbf{Marples B, Adomat H, Koch CJ, Skov KA (1996)}\\
\textit{Response of V79 cells to low doses of X-rays and negative pi-mesons: Clonogenic survival and DNA strand breaks.}\\
International Journal of Radiation Biology, 70(4):429-436.\\
PMID: 8862454\\
\url{https://pubmed.ncbi.nlm.nih.gov/8862454/}

\item \textbf{Wouters BG, Sy AM, Skarsgard LD (1996)} $\bigstar$ \textbf{ÉTUDE CLÉ SUR PANEL DE LIGNÉES}\\
\textit{Low-Dose Hypersensitivity and Increased Radioresistance in a Panel of Human Tumor Cell Lines with Different Radiosensitivity.}\\
Radiation Research, 146(4):399-413.\\
PMID: 8927712\\
\url{https://pubmed.ncbi.nlm.nih.gov/8927712/}\\
DOI: \url{https://doi.org/10.2307/3579302}\\
\textcolor{gray}{\textit{Note: Étude de 5 lignées tumorales humaines avec sensibilités variables. Les 4 lignées les plus résistantes montrent une hypersensibilité initiale aux faibles doses suivie d'une augmentation de radioresistance entre 0.3 et 0.7 Gy.}}

\item \textbf{Skarsgard LD, Skwarchuk MW, Wouters BG, Durand RE (1996)}\\
\textit{Substructure in the radiation survival response at low dose in cells of human tumor cell lines.}\\
Radiation Research, 146(4):388-398.\\
PMID: 8927711\\
\url{https://pubmed.ncbi.nlm.nih.gov/8927711/}

\item \textbf{Joiner MC, Lambin P, Malaise EP, Robson T, Arrand JE, Skov KA, Marples B (1996)} $\bigstar$ \textbf{REVUE MAJEURE}\\
\textit{Hypersensitivity to very-low single radiation doses: its relationship to the adaptive response and induced radioresistance.}\\
Mutation Research, 358(2):171-183.\\
PMID: 8946022\\
\url{https://pubmed.ncbi.nlm.nih.gov/8946022/}\\
\textcolor{gray}{\textit{Note: Revue établissant qu'une petite dose de conditionnement ($<$30 cGy) peut protéger contre une exposition ultérieure plus importante (réponse adaptative).}}

\item \textbf{Marples B, Lambin P, Skov KA, Joiner MC (1997)} $\bigstar$ \textbf{REVUE SYNTHÉTIQUE}\\
\textit{Low dose hyper-radiosensitivity and increased radioresistance in mammalian cells.}\\
International Journal of Radiation Biology, 71(6):721-735.\\
PMID: 9246186\\
\url{https://pubmed.ncbi.nlm.nih.gov/9246186/}\\
\textcolor{gray}{\textit{Note: Revue des travaux du Gray Laboratory (UK) et du BC Cancer Research Centre (Canada) sur l'HRS détectée après doses uniques de rayons X $<$0.3 Gy et la réponse IRR jusqu'à 1 Gy.}}

\item \textbf{Wouters BG, Skarsgard LD (1997)}\\
\textit{Low-dose radiation sensitivity and induced radioresistance to cell killing in HT-29 cells is distinct from the `adaptive response' and cannot be explained by a subpopulation of sensitive cells.}\\
Radiation Research, 148(5):435-442.\\
PMID: 9355868\\
\url{https://pubmed.ncbi.nlm.nih.gov/9355868/}

\end{enumerate}

%------------------------------------------------------------------------------
\subsection{Études sur lignées spécifiques (1999--2004)}
%------------------------------------------------------------------------------

\begin{enumerate}
\setcounter{enumi}{17}

\item \textbf{Short S, Mayes C, Woodcock M, Johns H, Joiner MC (1999)}\\
\textit{Low dose hypersensitivity in the T98G human glioblastoma cell line.}\\
International Journal of Radiation Biology, 75(7):847-855.\\
PMID: 10489896\\
\url{https://pubmed.ncbi.nlm.nih.gov/10489896/}\\
\textcolor{gray}{\textit{Note: T98G montre une HRS marquée, caractéristique de toute la population cellulaire plutôt que d'une sous-population hypersensible.}}

\item \textbf{Vaganay-Juéry S et al. (2000)}\\
\textit{Decreased DNA-PK activity in human cancer cells exhibiting hypersensitivity to low-dose irradiation.}\\
British Journal of Cancer, 83(4):514-518.\\
PMID: 10945500\\
\url{https://pubmed.ncbi.nlm.nih.gov/10945500/}\\
\textcolor{gray}{\textit{Note: Étude de 10 lignées cancéreuses humaines montrant une diminution marquée de l'activité DNA-PK dans les 6 lignées présentant HRS après irradiation à 0.2 Gy.}}

\item \textbf{Short SC, Kelly J, Mayes CR, Woodcock M, Joiner MC (2001)}\\
\textit{Low-dose hypersensitivity after fractionated low-dose irradiation in vitro.}\\
International Journal of Radiation Biology, 77(6):655-664.\\
PMID: 11403705\\
\url{https://pubmed.ncbi.nlm.nih.gov/11403705/}\\
\textcolor{gray}{\textit{Note: Étude de 4 lignées gliomes humains (T98G, U87, A7, U373). Trois lignées montrent HRS après doses uniques; U373 ne montre pas d'HRS.}}

\item \textbf{Joiner MC, Marples B, Lambin P, Short SC, Turesson I (2001)} $\bigstar$ \textbf{REVUE DE RÉFÉRENCE}\\
\textit{Low-dose hypersensitivity: current status and possible mechanisms.}\\
International Journal of Radiation Oncology Biology Physics, 49(2):379-389.\\
PMID: 11173131\\
\url{https://pubmed.ncbi.nlm.nih.gov/11173131/}\\
\textcolor{gray}{\textit{Note: La plupart des lignées cellulaires présentent une hyper-radiosensibilité (HRS) aux très faibles doses ($<$10 cGy) non prédite par rétro-extrapolation depuis les doses plus élevées.}}

\item \textbf{Chandna S, Dwarakanath BS, Khaitan D, Mathew TL, Jain V (2002)}\\
\textit{Low-dose radiation hypersensitivity in human tumor cell lines: effects of cell-cell contact and nutritional deprivation.}\\
Radiation Research, 157(5):516-525.\\
PMID: 11966317\\
\url{https://pubmed.ncbi.nlm.nih.gov/11966317/}\\
\textcolor{gray}{\textit{Note: Étude sur BMG-1, U-87 (gliomes) et PECA-4451, PECA-4197 (carcinomes squameux oraux).}}

\item \textbf{Tsoulou E, Baggio L, Cherubini R, Kalfas CA (2002)}\\
\textit{Radiosensitivity of V79 cells after alpha particle radiation at low doses.}\\
Radiation Protection Dosimetry, 99(1-4):237-240.\\
PMID: 12194294\\
\url{https://pubmed.ncbi.nlm.nih.gov/12194294/}\\
\textcolor{gray}{\textit{Note: Démonstration de l'HRS des cellules V79 après irradiation aux rayons gamma et ions $^{4}$He$^{2+}$ de différents LET.}}

\item \textbf{Short SC, Woodcock M, Marples B, Joiner MC (2003)}\\
\textit{Effects of cell cycle phase on low-dose hyper-radiosensitivity.}\\
International Journal of Radiation Biology, 79(2):99-105.\\
PMID: 12569013\\
\url{https://pubmed.ncbi.nlm.nih.gov/12569013/}\\
\textcolor{gray}{\textit{Note: Dans T98G, toutes les populations montrent l'HRS mais l'effet est le plus marqué en phase G2. Dans U373, l'HRS n'est démontrée que dans les cellules G2.}}

\item \textbf{Marples B, Wouters BG, Joiner MC (2003)}\\
\textit{An association between the radiation-induced arrest of G2-phase cells and low-dose hyper-radiosensitivity: a plausible underlying mechanism?}\\
Radiation Research, 160(1):38-45.\\
PMID: 12816521\\
\url{https://pubmed.ncbi.nlm.nih.gov/12816521/}\\
\textcolor{gray}{\textit{Note: Les cellules T98G et V79, qui montrent l'HRS, échouent à arrêter l'entrée en mitose des cellules G2 endommagées à des doses $<$30 cGy.}}

\item \textbf{Beauchesne PD et al. (2003)}\\
\textit{Human malignant glioma cell lines are sensitive to low radiation doses.}\\
International Journal of Cancer, 105(1):33-40.\\
PMID: 12672027\\
\url{https://pubmed.ncbi.nlm.nih.gov/12672027/}\\
\textcolor{gray}{\textit{Note: 4 des 5 lignées de gliomes humains étudiées montrent une sensibilité significative aux rayons X à des doses $<$1 Gy.}}

\item \textbf{Marples B, Wouters BG, Collis SJ, Chalmers AJ, Joiner MC (2004)} $\bigstar$ \textbf{MODÈLE MÉCANISTIQUE}\\
\textit{Low-dose hyper-radiosensitivity: a consequence of ineffective cell cycle arrest of radiation-damaged G2-phase cells.}\\
Radiation Research, 161(3):247-255.\\
PMID: 14982490\\
\url{https://pubmed.ncbi.nlm.nih.gov/14982490/}\\
\textcolor{gray}{\textit{Note: Proposition du modèle à trois composantes : reconnaissance des dommages, transduction du signal, réparation de l'ADN.}}

\item \textbf{Marples B (2004)}\\
\textit{Is low-dose hyper-radiosensitivity a measure of G2-phase cell radiosensitivity?}\\
Cancer and Metastasis Reviews, 23(3-4):197-207.\\
PMID: 15197323\\
\url{https://pubmed.ncbi.nlm.nih.gov/15197323/}

\end{enumerate}

%==============================================================================
\section{Publications Récentes (2022--2025)}
%==============================================================================

%------------------------------------------------------------------------------
\subsection{Études fondamentales}
%------------------------------------------------------------------------------

\begin{enumerate}

\item \textbf{Polgár S, Schofield PN, Madas BG (2022)}\\
\textit{Datasets of in vitro clonogenic assays showing low dose hyper-radiosensitivity and induced radioresistance.}\\
Scientific Data, 9(1):555.\\
PMID: 36075916\\
\url{https://www.nature.com/articles/s41597-022-01653-3}\\
DOI: 10.1038/s41597-022-01653-3\\
\textcolor{gray}{\textit{Note: Base de données de 46 publications avec 101 jeux de données sur les fractions de survie avec preuves d'HRS.}}

\item \textbf{Ma CMC (2022)}\\
\textit{Pulsed low dose-rate radiotherapy: radiobiology and dosimetry.}\\
Physics in Medicine \& Biology, 67(3).\\
PMID: 35038688\\
\url{https://pubmed.ncbi.nlm.nih.gov/35038688/}\\
\textcolor{gray}{\textit{Note: Revue complète sur les fondements radiobiologiques du PLDR.}}

\item \textbf{Khan AU, Radtke J, Hammer C et al. (2024)}\\
\textit{Dose-rate dependence and IMRT QA suitability of EBT3 radiochromic films for pulse reduced dose-rate radiotherapy (PRDR) dosimetry.}\\
Journal of Applied Clinical Medical Physics, 25(1):e14229.\\
PMID: 38032123\\
\url{https://pubmed.ncbi.nlm.nih.gov/38032123/}

\item \textbf{Kato TA et al. (2024)}\\
\textit{Exploring DNA repair deficient CHO cell response to low dose rate radiation.}\\
Biochemical and Biophysical Research Communications, 697:149547.\\
\url{https://www.sciencedirect.com/science/article/abs/pii/S0006291X24000743}\\
\textcolor{gray}{\textit{Note: Étude sur la réponse des cellules mutantes déficientes en réparation de l'ADN (NHEJ, HR, Fanconi Anemia, PARP).}}

\item \textbf{Franco-Barraza J et al. (2024)}\\
\textit{Pulsed low-dose-rate radiation (PLDR) reduces the tumor-promoting responses induced by conventional chemoradiation in pancreatic cancer-associated fibroblasts.}\\
bioRxiv (preprint), 2024.01.13.575510.\\
PMID: 38293200\\
\url{https://pubmed.ncbi.nlm.nih.gov/38293200/}\\
DOI: 10.1101/2024.01.13.575510

\item \textbf{Salem NR, Eldib A, El-Sayed EM, Mostafa E, Desouky OS (2024)}\\
\textit{Toxicity assessment following conventional radiation therapy and pulsed low dose rate radiation therapy: an in vivo animal study.}\\
Radiation Oncology, 19(1):159.\\
PMID: 39538311\\
\url{https://pubmed.ncbi.nlm.nih.gov/39538311/}

\end{enumerate}

%------------------------------------------------------------------------------
\subsection{Études cliniques et applications thérapeutiques}
%------------------------------------------------------------------------------

\begin{enumerate}
\setcounter{enumi}{6}

\item \textbf{Wong RX, Master Z, Pang E, Yang V, Looi WS (2024)}\\
\textit{Pulsed low-dose rate radiotherapy for recurrent bone sarcomas: case reports and brief review.}\\
Radiation Oncology Journal, 42(1):88-94.\\
PMID: 38549388\\
\url{https://pubmed.ncbi.nlm.nih.gov/38549388/}

\item \textbf{Investigation of Normal Tissue Toxicity in PLDR (2025)}\\
\textit{Investigation of Normal Tissue Toxicity in Pulsed Low Dose Rate Radiotherapy.}\\
PMID: 40427198\\
\url{https://pubmed.ncbi.nlm.nih.gov/40427198/}\\
\textcolor{gray}{\textit{Note: PLDR démontre une toxicité réduite par rapport à la radiothérapie conventionnelle.}}

\item \textbf{Atak et al. (2025)}\\
\textit{Pulsed reduced dose rate radiotherapy: a narrative review.}\\
Chinese Clinical Oncology.\\
\url{https://cco.amegroups.org/article/view/139535/html}

\item \textbf{Sevilya Z et al. (2025)}\\
\textit{Robotic radiation shielding system reduces radiation-induced DNA damage in operators performing electrophysiological procedures.}\\
Scientific Reports, 15:18603.\\
\url{https://www.nature.com/articles/s41598-025-03686-1}\\
DOI: 10.1038/s41598-025-03686-1

\end{enumerate}

%==============================================================================
\section{Lignées cellulaires étudiées historiquement}
%==============================================================================

\begin{table}[h!]
\centering
\begin{tabular}{llcl}
\toprule
\textbf{Lignée} & \textbf{Type} & \textbf{HRS} & \textbf{Références clés} \\
\midrule
V79 & Hamster chinois & $\checkmark$ Marquée & Marples \& Joiner 1993 \\
T98G & Glioblastome humain & $\checkmark$ Marquée & Short 1999, 2003 \\
U373 & Gliome humain & $\times$ (sauf G2) & Short 2001, 2003 \\
U87 & Glioblastome humain & $\checkmark$ & Short 2001 \\
A7 & Glioblastome humain & $\checkmark$ & Short 2001 \\
U138MG & Glioblastome humain & $\checkmark$ & Krueger 2013 \\
HT-29 & Colon humain & $\checkmark$ & Wouters 1997 \\
BMG-1 & Gliome humain & $\checkmark$ Marquée & Chandna 2002 \\
HeLa & Col utérus humain & $\checkmark$ & Diverses études \\
PC3 & Prostate humain & $\checkmark$ & Mitchell 2002 \\
A549 & Poumon humain & $\checkmark$ & Dai 2009 \\
MCF-7 & Sein humain & $\checkmark$ & Guirado 2012 \\
\bottomrule
\end{tabular}
\caption{Principales lignées cellulaires étudiées pour le phénomène HRS/IRR}
\end{table}

%==============================================================================
\section{Paramètres typiques du modèle Induced Repair}
%==============================================================================

Le modèle de réparation induite (IR) modifie l'équation linéaire-quadratique (LQ) standard:

\begin{equation}
SF = \exp\left[-\alpha D - \beta D^2\right]
\end{equation}

En introduisant une variation de $\alpha$ avec la dose:

\begin{equation}
\alpha(D) = \alpha_r + (\alpha_s - \alpha_r) \exp\left(-\frac{D}{D_c}\right)
\end{equation}

où:
\begin{itemize}
    \item $\alpha_s$ = pente initiale (hypersensibilité, $\sim$0.3--0.6 Gy$^{-1}$)
    \item $\alpha_r$ = pente à haute dose (résistance, $\sim$0.1--0.3 Gy$^{-1}$)
    \item $D_c$ = dose de transition ($\sim$0.2--0.4 Gy selon la lignée)
\end{itemize}

%==============================================================================
\section{Ressources supplémentaires}
%==============================================================================

\subsection{Base de données publique}
\begin{itemize}
    \item \textbf{STOREDB} -- Base de données HRS/IRR (Polgár et al. 2022)\\
    Contient 101 jeux de données de 46 publications (1993--2021)\\
    \url{https://www.nature.com/articles/s41597-022-01653-3}
\end{itemize}

\subsection{Essais cliniques en cours}
\begin{itemize}
    \item NCT03061162 -- PLDR pour tumeurs réfractaires
    \item NCT04452357 -- PLDR pour cancer du pancréas
\end{itemize}

\end{document}
