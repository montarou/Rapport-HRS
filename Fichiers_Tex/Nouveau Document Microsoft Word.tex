



\noindent 

\noindent 

\noindent 

\noindent 

\noindent 

\noindent 

\noindent 

\noindent 

\noindent 

\noindent 

\noindent 

\noindent 

\noindent 

\noindent 

\noindent 

\noindent 

\noindent 

\noindent 

\noindent 

\noindent 

\noindent 

\noindent 

\noindent 

\noindent 

\noindent 

\noindent 

\noindent 

\noindent 

\noindent 

\noindent 

\noindent 

\noindent 

\noindent 

\noindent 

\noindent 

\noindent 

\noindent 

\noindent Int J Radiat Biol 1993 May;63\eqref{GrindEQ__5_}:639-50.

\noindent Hypersensitivity of a human tumour cell line to very low radiation doses

\noindent P Lambin  1 , B Marples, B Fertil, E P Malaise, M C Joiner

\noindent 

\noindent Survival of HT29 cells was measured after irradiation with single doses of X-rays (0.05-5 Gy) and neutrons (0.025-1.5 Gy), using a Dynamic Microscopic Imaging Processing Scanner (DMIPS) with which individual cells can be accurately located in tissue culture flasks, their positions recorded, and after an appropriate incubation time the recorded positions revisited to allow the scoring of survivors. The response over the X-ray dose range 2-5 Gy showed a good fit to a Linear-Quadratic (LQ) model. For X-ray doses below 1 Gy, an increased X-ray effectiveness was observed with cell survival below the high-dose LQ prediction. The value of --dose/loge (SF) for each experimental data point, plotted against dose, demonstrated clearly how X-rays are maximally effective at doses approaching zero, becoming less effective as the dose increases and with minimal effectiveness at about 0.6 Gy then becoming more effective again as the dose increases above 1.5 Gy. This phenomenon was not seen with neutrons. Neutron RBE was calculated for each X-ray data point by taking each X-ray survival value and comparing it with the common LQ fit to all the neutron data. Over the X-ray dose range 0.05-0.2 Gy, the RBE is close to 1 indicating that these very low doses of X-rays are of similar effectiveness to neutrons in killing cells. The increase in RBE with increasing dose over the range 0.05-1 Gy, and the slight decrease in RBE above 1 Gy, reflect primarily the changes in X-ray sensitivity over the whole dose range of 0.05-5 Gy. Several arguments suggest that this phenomenon could reflect an induced radioresistance so that in this system low single doses of X-rays are more effective per Gy than higher doses in reducing cell survival because only at higher doses, above a threshold, is there sufficient damage to trigger radioprotective mechanisms.

\noindent 

\noindent Radiat Res . 1994 Apr;138(1 Suppl):S25-7. 

\noindent Multiphasic Survival Curves for Cells of Human of human cell lines to small doses. Are there some clinical implications?\textbf{ }

\noindent E P Malaise${}^{~}$,~P Lambin,~M C Joiner 

\noindent Survival of the cells of three human tumor cell lines of differing radiosensitivity was measured after irradiation with single doses of X rays (0.05-5 Gy). At doses below 1 Gy, cells were more radiosensitive than predicted by back-extrapolating the high-dose response. This difference was more marked for cells of the radioresistant cell lines than the radiosensitive cell line so that the "true" initial slopes of the survival curves, at very low doses, were similar for the cells of the three cell lines. This phenomenon could reflect an induced radioresistance so that low doses of X rays are more effective per gray than higher doses, because only at higher doses is there sufficient damage to trigger repair systems or other radioprotective mechanisms which can then act during the time course for repair of DNA injury.

\noindent 

\noindent 

\noindent Radiat Res. 1996 Oct;146\eqref{GrindEQ__4_}:399-413. 

\noindent Low-dose hypersensitivity and increased radioresistance in a panel of human tumor cell lines with different radiosensitivity 

\noindent B G Wouters${}^{~}$,~A M Sy,~L D Skarsgard 

\noindent 

\noindent It is well known that cells of human tumor cell lines display a wide range of sensitivity to radiation, at least a part of which can be attributed to different capacities to process and repair radiation damage correctly. We have examined the response to very low-dose radiation of cells of five human tumor cell lines that display varying sensitivity to radiation, using an improved assay for measurement of radiation survival. This assay improves on the precision of conventional techniques by accurately determining the numbers of cells at risk, and has allowed us to measure radiation survival to doses as low as 0.05 Gy. Because of the statistical limitations in measuring radiation survival at very low doses, extensive averaging of data was used to determine the survival response accurately. Our results show that the four most resistant cell lines exhibit a region of initial low-dose hypersensitivity. This hypersensitivity is followed by an increase in radioresistance over the dose range 0.3 to 0.7 Gy, beyond which the response is typical of that seen in most survival curves. Mathematical modeling of the responses suggests that this phenomenon is not due to a small subpopulation of sensitive cells (e.g. mitotic), but rather is a reflection of the induction of resistance in the whole cell population, or at least a significant proportion of the whole cell population. These results suggest that a dose-dependent alteration in the processing of DNA damage over the initial low-dose region of cell survival may contribute to radioresistance in some cell lines. 

\noindent 

\noindent 

\noindent 

\noindent 

\noindent Int J Radiat Oncol Biol Phys . 2001 Feb 1;49\eqref{GrindEQ__2_}:379-89. 

\noindent Low-dose hypersensitivity: current status and possible mechanisms 

\noindent M C Joiner${}^{~}$,~B Marples,~P Lambin,~S C Short,~I Turesson 

\noindent \underbar{}

\noindent To retain cell viability, mammalian cells can increase damage repair in response to excessive radiation-induced injury. The adaptive response to small radiation doses is an example of this induced resistance and has been studied for many years, particularly in human lymphocytes. This review focuses on another manifestation of actively increased resistance that is of potential interest for developing improved radiotherapy, specifically the phenomenon in which cells die from excessive sensitivity to small single doses of ionizing radiation but remain more resistant (per unit dose) to larger single doses. In this paper, we propose possible mechanisms to explain this phenomenon based on our data accumulated over the last decade and a review of the literature. 

\noindent \textbf{Conclusion: }Typically, most cell lines exhibit hyper-radiosensitivity (HRS) to very low radiation doses ($\mathrm{<}$10 cGy) that is not predicted by back-extrapolating the cell survival response from higher doses. As the dose is increased above about 30 cGy, there is increased radioresistance (IRR) until at doses beyond about 1 Gy, radioresistance is maximal, and the cell survival follows the usual downward-bending curve with increasing dose. The precise operational and activational mechanism of the process is still unclear, but we propose two hypotheses. The greater amount of injury produced by larger doses either \eqref{GrindEQ__1_} is above a putative damage-sensing threshold for triggering faster or more efficient DNA repair or \eqref{GrindEQ__2_} causes changes in DNA structure or organization that facilitates constitutive repair. In both scenarios, this enhanced repair ability is decreased again on a similar time scale to the rate of removal of DNA damage. 

\noindent 

\noindent 

\noindent Radiat Res. 2004 Mar;161\eqref{GrindEQ__3_}:247-55. 

\noindent Low-dose hyper-radiosensitivity: a consequence of ineffective cell cycle arrest of radiation-damaged G2-phase cells 

\noindent B Marples${}^{~}$,~B G Wouters,~S J Collis,~A J Chalmers,~M C Joiner 

\noindent 

\noindent This review highlights the phenomenon of low-dose hyper- radiosensitivity (HRS), an effect in which cells die from excessive sensitivity to small single doses of ionizing radiation but become more resistant (per unit dose) to larger single doses. Established and new data pertaining to HRS are discussed with respect to its possible underlying molecular mechanisms. To explain HRS, a three-component model is proposed that consists of damage recognition, signal transduction and damage repair. The foundation of the model is a rapidly occurring dose-dependent pre-mitotic cell cycle checkpoint that is specific to cells irradiated in the G2phase. This checkpoint exhibits a dose expression profile that is identical to the cell survival pattern that characterizes HRS and is probably the key control element of low-dose radiosensitivity. This premise is strengthened by the recent observation coupling low- dose radiosensitivity of G2-phase cells directly to HRS. The putative role of known damage response factors such as ATM, PARP, H2AX, 53BP1 and HDAC4 is also included within the framework of the HRS model. 

\noindent 

\noindent 

\noindent 


\end{document}

